% !TEX root = ../Vorlage_DA.tex
%	########################################################
% 							Zusammenfassung
%	########################################################


%	--------------------------------------------------------
% 	Überschrift, Inhaltsverzeichnis
%	--------------------------------------------------------
\chapter*{Summary}
\addcontentsline{toc}{chapter}{Summary}

%	--------------------------------------------------------
% 	Inhalt
%	--------------------------------------------------------
This paper starts with the introduction into Simultaneous Localization and Mapping (\gls{slam}) in general and two specific methods: ORB-SLAM and LSD-SLAM in detail. \newline
Chapter 2 explains the Robot Operating System (\gls{ros}), which provides the platform for the different \gls{slam} methods. Chapter 3 introduces Artificial Neural Networks (ANN). This Chapter explains the components and also the most important types of Artificial Neural Networks. Feed Forward Neural Networks and Convolution Neural Networks. The next step is Deep Learning. There are three different ways it can be implemented. Supervised, Semi-Supervised and Unsupervised. All of this methods need camera data. Therefore the Basler camera is explained . \newline
After the basics a more detailed look into the ORB-SLAM is taken to make it easier to understand the concept of the subsequent topic. The topics of the ORB-SLAM and LSD-SLAM include how the algorithm works, what data they require and in which way they can be used. Another art in this section is the DeepTAM and how it is structured and how it works. \newline
Last Section is the Workflow where the used hardware and software is listed, how it was setup, how the errors where handled that occurred during the project and how everything was used.