% !TEX root = ../Vorlage_DA.tex
%	########################################################
% 							Zusammenfassung
%	########################################################


%	--------------------------------------------------------
% 	Überschrift, Inhaltsverzeichnis
%	--------------------------------------------------------
\chapter*{Summary}
\addcontentsline{toc}{chapter}{Summary}

%	--------------------------------------------------------
% 	Inhalt
%	--------------------------------------------------------
This paper starts with the introduction into the required basic before going on with the used methods and how they work. \newline
The basics starts with the general principle about  \gls{slam}s, what they do and how they work. Next is the software which handles the communication between the individual algorithms called ROS and the most important tools that it comes with.. Afterwards there will be a detailed view into the structure and use of Artificial Neural Networks and what Deep Learning means. A smaller part is the explanation of the Basler camera which sits on the car the system is running on. \newline
After the basics a more detailed look into the ORB-SLAM is taken to make it easier to understand the concept of the subsequent topic. The topics of the ORB-SLAM and LSD-SLAM include how the algorithm works, what data they require and in which way they can be used. Another art in this section is the DeepTAM and how it is structured and how it works. \newline
Last Section is the Workflow where the used hardware and software is listed, how it was setup, how the errors where handled that occurred during the project and how everything was used.