% !TEX root = ../Vorlage_DA.tex
%	########################################################
% 							Zusammenfassung
%	########################################################


%	--------------------------------------------------------
% 	Überschrift, Inhaltsverzeichnis
%	--------------------------------------------------------
\chapter*{Summary}
\addcontentsline{toc}{chapter}{Summary}

%	--------------------------------------------------------
% 	Inhalt
%	--------------------------------------------------------
This paper starts with the introduction into Simultaneous Localization and Mapping (\gls{slam}) in general and the ORB-\gls{slam} and LSD-\gls{slam} in more detail. \newline
Both methods need camera data. Therefore, the Basler camera is explained in chapter \ref{ref:baslercamera}. This image data needs to be collected and transferred to the \gls{slam} methods. For this task the Robot Operating System (\gls{ros}) is used. Chapter \ref{ref:ros} explains \gls{ros}, which provides the platform for the different \gls{slam} methods. \gls{ros} is a publish/subscribe based System. In chapter \ref{ref:orbslam} a more detailed look into the ORB-SLAM is taken to make it easier to understand the concept of the LSD-\gls{slam} in chapter \ref{ref:lsdslam}. The topics of the ORB-\gls{slam} and LSD-\gls{slam} include how the algorithm work, what data they require and in which way they can be used. The result are the current position and a pointcloud which can be transferred into \gls{ros} which has a tool called Rviz \ref{rviz} to visualize the pointcloud with the current position in it. \newline
In contrast to the first two methods, DeepTAM works with Artificial Intelligence. Therefore, Artificial Neural Networks (\gls{ann}) are needed which are explained in chapter \ref{ref:ann}. This chapter explains the components and the most important types of Artificial Neural Networks for DeepTAM. Feed Forward Neural Networks and Convolution Neural Networks. The next chapter is Deep Learning. There are three different ways Deep Learning can be implemented. Supervised, Semi-Supervised and Unsupervised. This all leads to chapter \ref{ref:deeptam} where DeepTAM itself is explained. \newline
Chapter \ref{ref:workflow} is the Workflow where the used hardware and software is listed, how it was setup, how the errors where handled that occurred during the project and how everything was used.
