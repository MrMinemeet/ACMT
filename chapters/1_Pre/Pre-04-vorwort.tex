% !TEX root = ../Vorlage_DA.tex
%	########################################################
% 							Vorwort
%	########################################################


%	--------------------------------------------------------
% 	Überschrift, Inhaltsverzeichnis
%	--------------------------------------------------------
\chapter*{Abstract}
% ENGLISH
An autonomous driving car has to orientate in the immediate vicinity. For this purpose the car needs a map and the own position in the map. Why is this such an important task when we already have a technology like Global Positioning System (\gls{gps})? \gls{gps} provides routing data, but no live details what happens on the street, like an ongoing construction or a pedestrian crossing the street. To get this information the car has to collect visual data from its environment in real-time. This thesis examines two approaches of visual data processing to generate a map and locate the car inside this map. One approach is Simultaneous Localization And Mapping  (\gls{slam}), the other one uses artificial intelligence. The platform for ORB-\gls{slam} and LSD-\gls{slam} is the Robot Operating System (\gls{ros}), which handles the publish/subscriber based communication between the camera and the \gls{slam} algorithms.\newline
DeepTAM uses Deep Neural Networks which can learn to predict the distance between objects in an image. The prediction is based on training data. The accuracy of the result depends on the amount of training data. It is a system for keyframe-based dense camera tracking and depth map estimation that is entirely learned. This thesis shows how to implement the different algorithms in \gls{ros} and points out the pros and cons of each method compared to the others and gives an outlook into future methods.
\newline
\newline
\newline
\newline
% GERMAN
Ein selbstfahrendes Auto muss sich in der unmittelbaren Umgebung orientieren. Dazu benötigt das Auto eine Karte der Umgebung und die eigene Position. Warum ist dies so eine wichtige Aufgabe, wenn wir bereits über Technologien wie das Global Positioning System (\gls{gps}) verfügen? GPS liefert zwar die Standortdaten, aber keine aktuellen Details darüber, was auf der Straße passiert, wie z.B. eine Baustelle oder ein Fußgänger der die Straße überquert. Um diese Informationen zu erhalten, muss das Auto visuelle Daten aus seiner Umgebung in Echtzeit sammeln. In dieser Arbeit werden zwei Ansätze der visuellen Datenverarbeitung untersucht, um eine Karte zu erstellen und das Auto in dieser Karte zu lokalisieren. Ein Ansatz ist Simultaneous Localization And Mapping (\gls{slam}), der Andere verwendet eine künstliche Intelligenz. Die Plattform für den ORB-SLAM und den LSD-SLAM ist das Robot Operating System (ROS), das die Publisher/Subscriber-basierte Kommunikation zwischen der Kamera und den SLAM-Algorithmen übernimmt. Der DeepTAM verwendet ein Deep Neural Network, welches trainiert wurde den Abstand zwischen Objekten in einem Bild vorherzusagen. Die Genauigkeit der Vorhersage beruht auf der Anzahl der Trainingsdaten. Der DeepTAM verwendet ein angelerntes System welches mit Keyframe-basiertem Dense Camera Tracking und Tiefenschätzung eine Karte generiert. Diese Arbeit zeigt die Funktionsweise der verschiedenen Algorithmen, beleuchtet deren Vor- und Nachteile und gibt einen Ausblick auf zukünftige Methoden.

 

\addcontentsline{toc}{chapter}{Abstract}