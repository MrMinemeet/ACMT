% !TEX root = ../Vorlage_DA.tex
%	########################################################
% 							Vorwort
%	########################################################


%	--------------------------------------------------------
% 	Überschrift, Inhaltsverzeichnis
%	--------------------------------------------------------
\chapter*{Abstract}
A big problem for self-driving cars is to analyze the environment it is in and how to process the incoming data to get the most information out of it.  Why is this such an important task when we already have technology like GPS? It is because GPS is not always that accurate or not available at all and also the information must be up to date all the time (E.g. changes due to a construction). This paper focuses on this problem when the input data is fetched by a camera and thus only contains visual data. There were three different ways tested. They all have in common that they use the data to generate a digital map and show the position off the car itself in this map. The first two methods ORB-SLAM and LSD-SLAM are mathematical algorithms. The Third method DeepTAM uses Deep Neural Networks which can learn adapt when given enough training data. 

\addcontentsline{toc}{chapter}{Abstract}


%	--------------------------------------------------------
% 	Inhalt
%	--------------------------------------------------------

Im Vorwort teilt der Bearbeiter dem Leser wichtige Tatsachen mit, die Erklärungen zu seiner Arbeit beinhalten -- z.B. die Motivation für die Bearbeitung des Themas oder besondere Schwierigkeiten bei der Bearbeitung und/oder Materialbeschaffung. 

Hier können auch Mitteilungen persönlicher Natur enthalten sein -- z.B. Dank an Institutionen/Personen für die geleistete Unterstützung. 