% !TEX root = ../Vorlage_DA.tex
%	########################################################
% 							Arbeitstitel
%	########################################################


%	--------------------------------------------------------
% 	Überschrift, Inhaltsverzeichnis
%	--------------------------------------------------------
\chapter*{Thema: \newline \htlArbeitsthema }



%	--------------------------------------------------------
% 	Bearbeiter
%	--------------------------------------------------------
\section*{Subtopics and Editor:}


\textbf{Implementing SLAMS and DeepTAM, Image Pre-Processing}\\ 
Alexander Voglsperger, 5AHELS\\
\emph{Advisors:} Dipl. Ing. Müller Gerhard\\[2ex] 
%
\textbf{Implementing DeepTAM, Gathering Trainingdata}\\ 
Simon Moharitsch, 5AHELS\\
\emph{Advisors:} Dipl. Ing. Müller Gerhard\\[2ex]



%	--------------------------------------------------------
% 	Beteiligte Firmen
%	--------------------------------------------------------
\section*{Projectpartner:}

\renewcommand{\arraystretch}{1.5}
\begin{tabularx}{1\textwidth}{@{} l X @{}}

\emph{Designation:} & Johannes Kepler University - Artificial Intelligence Lab\\
\emph{Address:} & Altenberger Straße 69\\
\emph{ZIP, location:} & 4040 Linz, Austria\\
\emph{Contact person:} & Dr. Nessler Bernhard\\
\emph{Phone:} & +43 (0)732 2468 4539\\
\emph{E-Mail:} & nessler@ml.jku.at\\

\end{tabularx}


%--------------------------------------------------------------------------------
%  Vorgeschriebene Dokumentationsseiten
%--------------------------------------------------------------------------------

\pagebreak
\thispagestyle{empty}
\newgeometry{top=2cm, bottom=1.5cm}

\begin{minipage}[c]{0.20\linewidth}
\includegraphics[width=0.8\linewidth]{media/images/htl_c_cmyk_rein}
\end{minipage}
\begin{minipage}[c]{0.6\linewidth}
\begin{center}
{\bfseries\sffamily\large Höhere  technische  Bundeslehranstalt\\
und  Bundesfachschule  Braunau\\
Elektronik und Technische Informatik\\
{\normalsize School autonomous focus on Mobile Computing and Software Engineering} }
\end{center}
\end{minipage}
\begin{minipage}[c]{0.2\linewidth}
\hfill \includegraphics[width=0.8\linewidth]{media/images/htl-bildung-mit-zukunft}
\end{minipage}\\

\vspace{1em}
\begin{center}
\bfseries\sffamily\Large
DIPLOMA DOCUMENTATION
\end{center}
\vspace{1ex}

\renewcommand{\arraystretch}{2}
\begin{tabularx}{1\textwidth}{ p{3.5cm} X }

\textbf{Author} & 
Alexander Voglsperger, Simon Moharitsch \\

\textbf{Vintage\linebreak School year} & 
5AHELS 2019/2020 \\

\textbf{\mbox{Topic of the } \mbox{diploma documentation}} & 
\htlArbeitsthema \\

\textbf{Cooperation\-partner} &
Johannes Kepler University - Artificial Intelligence Lab\\

\textbf{Task definition} & 
{The task of this work is to get depth information out of a video stream in real time. The considered field of application is an autonomous driving car that uses several cameras to orientate itself while driving. In a first approach the Robot Operating System (\gls{ros}) sends the video of a camera to two Simultaneous Localization And Mapping (\gls{slam}) algorithms which gather depth information out of the images. This results in a point cloud which represents the detected surroundings. Another approach is to get the DeepTAM, a method which implements artificial intelligence methods to estimate distances between objects in each frame of a video stream. As DeepTAM has not been updated for a while there are many compatibility issues with newer drivers, newer Tensor Flow framework and other required libraries. Thus the task is to fix the compatibility issue.}\\

\textbf{Realization} & 
{Simultaneous Localization And Mapping (\gls{slam}) algorithms are implemented on the free available platform Robot Operating System (\gls{ros}). ORB-\gls{slam} and LSD-\gls{slam} are two specific SLAM methods. These two \gls{slam}s have been modified to get executable program code on \gls{ros}. Several performance tests and a comparison between ORB-\gls{slam} and LSD-\gls{slam} are also part of this work. } \\

\textbf{Outcome} & 
{In this thesis the results of ORB-\gls{slam} and the LSD-\gls{slam} are compared. The ORB-\gls{slam} is working, but will not produce a detailed map if prominent points are missing in the video. Prominent points are characterized by sharp edges or corners of an object. \gls{slam} methods use these prominent points to get the depth information in a 2D video stream. The LSD-\gls{slam} does not work as good as the ORB-\gls{slam} when the camera only has a axial movement in the sense that it requires both axial and rotational movement to work and thus would generate an incorrect map. Otherwise the LSD-\gls{slam} only works on cars when a wide-angle camera lens is used. DeepTAM works on Ubuntu 16.04. Since this is not up to date, most libraries need to be downgraded, which creates driver problems. To get DeepTAM installed on the current version of Ubuntu, the source code of DeepTAM has to be adapted.} \\

\end{tabularx}

%--------------------------------------------------------------------------------

\pagebreak
\thispagestyle{empty}
\newgeometry{top=2cm, bottom=1.5cm}


\begin{minipage}[c]{0.20\linewidth}
\includegraphics[width=0.8\linewidth]{media/images/htl_c_cmyk_rein}
\end{minipage}
\begin{minipage}[c]{0.6\linewidth}
\begin{center}
{\bfseries\sffamily\large Höhere  technische  Bundeslehranstalt\\
und  Bundesfachschule  Braunau\\
Elektronik und Technische Informatik\\
{\normalsize School autonomous focus on Mobile Computing and Software Engineering} }
\end{center}
\end{minipage}
\begin{minipage}[c]{0.2\linewidth}
\hfill \includegraphics[width=0.8\linewidth]{media/images/htl-bildung-mit-zukunft}
\end{minipage}\\

\vspace{1em}

\renewcommand{\arraystretch}{2}
\begin{tabularx}{1\textwidth}{ p{3.5cm} X }

\textbf{\mbox{Illustrative graph,} \mbox{photo} \mbox{(incl. explanation)}} & 
{
Structure of the flowchart:
\newline
\newline
\begin{center}
	\includegraphics[width=1\linewidth]{media/images/illustrative_graph.png}
\end{center}
} \\
  
%\textbf{\mbox{Participation in} competitions, Awards} & 
%{
%Participated:
%\begin{itemize}
%\item Jugend Innovativ
%\item ITs Award
%\item FH Kärnten Maturaprojekt-Wettbewerb
%\item computer creative wettbewerb
%\item 120 Sekunden Wettbewerb
%\end{itemize}
%} \\

\textbf{\mbox{Accessibility of} \mbox{diploma thesis}} & 
{HTL Braunau archive, or\newline \url{https://diplomarbeiten.berufsbildendeschulen.at/}} \\



\end{tabularx}




%--------------------------------------------------------------------------------
% Unterschriften
%--------------------------------------------------------------------------------



\vspace*{\fill}

\textbf{Approval (date / signature)}

\fbox{
\begin{minipage}[t][3cm]{0.5\linewidth}
\centering
    Examiner \\
    %\hspace*{\fill}\includegraphics[width=0.8\linewidth]{fig/Unterschrift}\hspace*{\fill}
    \vfill
    \end{minipage}}
    \fbox{
    \begin{minipage}[t][3cm]{0.5\linewidth}
    \centering
    Head of College / Department
    \vfill
\end{minipage}
}

\restoregeometry

