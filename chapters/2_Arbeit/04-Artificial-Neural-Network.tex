% Discription of what is a Artificial Neural Networks and what does it
\chapter{Artificial Neural Networks\authorB}

\section{What is a Artificial Neural Networks?}

  \textbf{A}rtificial \textbf{N}eural \textbf{N}etworks(ANN) are inspired by biological neural networks that constitute animal brains. Important to notice is that they are not faithful models of biologic neural or cognitive phenomena. In fact most of these models are more closely related to mathematical and/or statistical models(For Example: clustering algorithms). Such systems "learn" to perform tasks by considering examples, generally without being programmed with task-specific rules. 
 
\section{Areas of Application}

 ANN are viable computational models for a wide variety of problems, including pattern classification, speech synthesis and recognition, adaptive interfaces between human and complex physical system, function approximation, associative memory, clustering, forecasting and prediction, combinatorial optimization, nonlinear system modeling, and control
 \cite{fundamentals_ann}
 
\section{Components of an ANN}

Simplified a ANN consists of three main components(neurons, connection and the weight associated with them) the propagation function and a bias. In the following topics I will give you a short summary what these components are and afterward I will explain how they work together. 

\subsection{Neurons}

Neurons are elementary units in an ANN. A neuron gets one ore more inputs and depending on the value of the inputs the output is set. A neuron can get its inputs from other neurons or, if its at the beginning, from the source of the data that needs to be processed. Depending on the Type of ANN they are placed in different structures. In most cases the output of a Neuron is a number between 0 and 1.

\begin{figure}[h]
	\centering
	\includegraphics[width=0.7\textwidth]{./media/images/diagram-for-general-view-of-artificial-neuron.jpg}
  	\caption{General view of Neurons
  	\\Source: https://tinyurl.com/yyfthk7c}
  	\label{Gvon}
\end{figure}

\subsection{Connection and weights}

These Neurons are Connected. Which neurons are connected with others depends on the structure. The weights characterize how important a connection between neurons is. 

\textbf{For example:}\newline
A neuron, we call it base for this example, has two neurons connected to it as inputs. The weight of the connection of the first neuron has a bigger weight than the second connection. That means the output of the base depends more on the input of the first neuron 

\subsection{Propagation function  and activation function}

This is a function witch takes the Inputs of a neuron, the weight of these connections and the bias and adds them up. The resulting value is the processed by the activation function which sets the output. One of the most common activation function is the sigmoid function because it is not a step function wich means the output doesn’t change instantaneously. Thats important for the training algorithmus.

\subsection{Bias}

The bias is a Neuron which has no Inputs. A bias is used to shift the decision boundary to the left or right.

\section{Organization}

A artificial neural network can be organized in many different ways.

\subsection{Feed Forward ANN}

\begin{figure}[h]
	\centering
	\includegraphics[width=0.7\textwidth]{./media/images/feed_forward_neural_network.png}
  	\caption{Feed forward network
  	\\Source: https://tinyurl.com/y3yaqboq}
  	\label{ffNN}
\end{figure}

The following picture demonstrates a feed forward ANN .There are a variable number of hidden layers depending on the purpose of the neural network. Nothing in the hidden layer is visible. 

\subsection{CNN}










